\chapter{Conclusion}
\lhead{\emph{Conclusion}}



Objectives of the project were to practically understand machine learning algorithms that could be used for classifying images, find a suitable food image dataset, preprocess the images, build image classifiers using various machine learning algorithms, train these classifiers with the accumulated dataset and evaluate the performance.

\section{Limitations}
The main limitation of the project was that machine learning algorithm takes a very long time to train. Images are very high dimensional data, and an enormous amount of the computing power is required to train a good machine learning model for image classification. Therefore, this project was limited to using only four image classes.  These four classes are not able to represent all types of food that are eaten by people.

The second limitation that was faced during this project was that it was not possible to explore larger deep learning networks on the hardware setup that was used. With deeper  neural networks it would be plausible to get  better  accuracy scores, then the accuracy scores received

\section{Future Work}

For extending this project, it should be focussed on one classification algorithm and making it work on a larger number of classes.  A convolutional neural network would be the most promising model to explore further because it achieved high accuracy score.  It would be possible to use a GPU cluster to train a model that could classify between multiple categories of food. There is not much research that has been done to explore convolutional neural networks for food classification. The classification models that were built classify only between four classes of pictures. This number of food categories is too small to use these classifiers for dietary assessment.  A good number of classes for a new model would be around 100 classes of most popular dishes. If this new model achieved the high accuracy rate, it would be possible to use this model in dietary assessment applications.

\section{Lessons Learnt}

During the progress of this project, it was gradually discovered why image classification is such a hard problem. Computer Vision compared to other subfields of AI such as voice recognition, seems to be progressing slowly.  The problem that algorithms face in image recognition is that images are very dynamic and are susceptible to different kind of variations. It was practically observed that machine learning algorithms that are used in practice for tasks such as spam detection, credit card fraud detection and other tasks were unable to classify images with high enough accuracy.  It was also observed that machine learning algorithms take a very long time to train. That could be a barrier to the adaptation of machine learning techniques. Overall, the project was successful. A framework for converting pictures to the dataset of images was built, different classification algorithms were successfully tried and evaluated.
