\chapter{Conclusions}
\lhead{\emph{Conclusions}}



The Objectives of the project were to practically understand machine learning algorithms that can be used for classifying images. To accomplish this task,  a suitable food image dataset was found and food pictures were preprocessed. Then, image classifiers using various machine learning algorithms were built.  Finally, these classifiers were trained on the accumulated dataset and their performances were evaluated.

\section{Lessons Learnt}

During the progress of this project, it was gradually discovered why image classification is such a hard problem.  The problem that algorithms in image recognition task face is that images are very dynamic and are susceptible to different kind of variations. It was practically observed that machine learning algorithms that are successfully used in practice for tasks such as spam detection, credit card fraud detection and other tasks were unable to classify images with a sufficient accuracy.  It was also observed that machine learning algorithms take a very long time to train. This is the main barrier for using machine learning techniques for vision. Overall, the project was successful. A framework for converting pictures to the dataset of images was built. Then,  machine learning classifiers that used different classification algorithms were developed and their performance was systematically evaluated.

\section{Limitations}
The main limitation of the project was that machine learning algorithms take a very long time to train. Images are very high dimensional data, and an enormous amount of the computing power is required to train a good machine learning model for image classification. Therefore, this project was limited to using only four image classes. These four classes are not able to represent all types of food that is eaten by people.

The second limitation that was faced during this project was that it was not possible to explore larger deep learning networks on the hardware setup used. Therefore, a maximum accuracy score of the deep neural network and the convolutional neural network could not be reached. With deeper architectures of neural networks, it would be plausible to achieve better accuracy scores.

\section{Future Work}

To extend this project, one classification method should be explored more thoroughly and a classifier that distinguishes between a greater amount of food classes should be built using this model.   A convolutional neural network would be the most promising model to explore further because it achieved a high accuracy score. Also, researchers have proven that a convolutional neural network is the best method that can be used for image classification. Furthermore, there here has been few studies about  a food classification with convolutional neural networks.  A good number of classes for a new model would be around 100 types of most popular dishes.  Because of the increased complexity, a GPU cluster will be needed to train this model. If this new model achieved the high accuracy rate, it would be possible to use this model in dietary assessment applications.


